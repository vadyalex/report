\subsection{Cluster Analysis Results Visualization}
\label{sec:cluster}

Figure~\ref{fig:Cytoscape_Cluster_2} from Section~\ref{sec:dataset_description} shows cluster graph specific structure: it is very high, unbalanced and has not so deep sub-parts. It is possible to use this disadvantage as advantage and abstract sub-parts to reduce drawing area. Extract those nodes and edges that form the longest path of the cluster graph - ,,backbone''. Figure~\ref{fig:cluster_visualisation} shows the algorithm step by step. Backbone vertices are filled with yellow and shown in Figure~\ref{fig:cluster_visualisation_algorithm_1}. Next step is to abstract branches into groups, group size is scaled according to amount of elements inside.

\begin{figure}[h!]
\centering
\subfloat[Backbone and branches]{
    \includegraphics[scale=0.15]{pictures/cluster_visualisation_algorithm_1.png}
    \label{fig:cluster_visualisation_algorithm_1}
}
\subfloat[Abstract branches into groups]{
    \includegraphics[scale=0.15]{pictures/cluster_visualisation_algorithm_2.png}
    \label{fig:cluster_visualisation_algorithm_2}
}
\subfloat[Scale group size]{
    \includegraphics[scale=0.15]{pictures/cluster_visualisation_algorithm_3.png}
    \label{fig:cluster_visualisation_algorithm_3}
}
\caption{Cluster Visualization algorithm}
\label{fig:cluster_visualisation_algorithm}
\end{figure}

The last step is to represent backbone as a spiral, thus preserving space and giving a possibility to show the complete tree in one view. Figure~\ref{fig:cluster_visualisation_algorithm_4} shows how does it work.

\begin{figure}[h!]
\centering
\includegraphics[scale=0.5]{pictures/cluster_visualisation_algorithm_4.png}
\caption{,,Rectangular Spiral Layout''}
\label{fig:cluster_visualisation_algorithm_4}
\end{figure}

Then backbone formed as rectangular spiral with a root in the center and moving in clockwise direction. Figure~\ref{fig:cluster_visualisation} shows complete visualization result for the real cluster tree. This will have to reuse space as much as possible and still gives overview of location of the highlighted vertices in cluster hierarchy -- how far from a root.

\begin{figure}[h!]
\centering
\includegraphics[scale=0.4]{pictures/cluster_spiral_visualisation.png}
\caption{Rectangular spiral Cluster graph visualization}
\label{fig:cluster_visualisation}
\end{figure}

It is possible to explore sub-parts (rectangles) of the Cluster graph using ,,lens''. User can interactively choose any sub-part and the lens with inner content will appear. There are two different lens layouts: polar (Figure~\ref{fig:lens_polar}) and HV-tree (Figure~\ref{fig:lens_tree}). Polar lens layout is based on the algorithm used for initial visualization of the Cluster graph, the algorithm was explained earlier in Section~\ref{sec:probe}. Both implementations are own made and are not based on any third party source code.

\begin{figure}[h!]
\centering
\subfloat[Polar lens layout]{
    \includegraphics[scale=0.5]{pictures/lens_polar.png}
    \label{fig:lens_polar}
}
\\
\subfloat[HV-tree lens layout]{
    \includegraphics[scale=0.5]{pictures/lens_tree.png}
    \label{fig:lens_tree}
}
\caption{Different lens layouts}
\end{figure}

