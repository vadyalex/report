\section*{Abstract}

The purpose of the thesis is to develop a new visualization method for Gene Ontologies and hierarchical clustering.
These are both important tools in biology and medicine to study high-throughput data such as transcriptomics and metabolomics data.
Enrichment of ontology terms in the data is used to identify statistically overrepresented ontology terms,
that give insight into relevant biological processes or functional modules.
Hierarchical clustering is a standard method to analyze and visualize data to find relatively homogeneous clusters of experimental data points.
Both methods support the analysis of the same data set, but are usually considered independently.
However, often a combined view such as: visualizing a large data set in the context of an ontology under consideration of a clustering of the data.
The result of the current work is a user-friendly program that combines two different views for analysing Gene Ontology and
Cluster simultaneously. To make explorations of such a big data possible we developed new visualization approach.

\section*{Keywords}
\label{sec:keywords}

\begin{itemize}
\item {\bf GO} \\ Gene Ontology;
\item {\bf XML} \\ eXtensible Markup Language;
\item {\bf GML} \\ Graph Modeling Language;
\item {\bf SVG} \\ Scalable Vector Graphics;
\item {\bf IDE} \\ Integrated Development Environment;
\item {\bf JUNG} \\ Java Universal Network / Graph Framework;
\item {\bf JOGL} \\ Java OpenGL;
\item {\bf JNI} \\ Java Native API;
\item {\bf LWJGL} \\ Lightweight Java Game Library;
\end{itemize}