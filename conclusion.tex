\section{Conclusion and Future Work}
\label{sec:conclusion}

We presented a new method for the combined visualization of an ontology (represented as DAG) and an hierarchical clustering (represented as tree) of one data set. The proposed method interactively visualizes all the data without scrolling, thereby presenting an complete overview. It also allows for interactive selection and navigation to explore the data. We have showed that the program is able to tackle the problem in our research focus, i.e., the visualization and visual mapping between two huge and conceptually different data sets. However, there are some improvements that should be performed in the future.

As explained previously, the zoomed-in view for the GO shows three levels at the same time while displaying the subgraph by highlighting the nodes only. The edges are omitted due to clutter problems that can occur since edges from a higher level might go through the zoomed-in view to nodes in the lower layers. Since we are zoomed in, this does no make sense to show, because we have no insight from which layer those edges are coming from, nor to which layer they are going to. However, an improvement is possible by showing only edges between the three layers shown in the zoomed-in GO view. At the same time, the edge bundling algorithm could also be improved.

Thesis application is not general visualization tool. It has been developed specially for biology scientist and driven by their requirements which means that the application fulfills their needs. There are no application with same functionality. Visualization technique developed during the work is based on provided data, uses its specific structure as advance. On the other hand the visualization approach may apply to another fields.

As for future improved version of the rectangular spiral metaphor it is possible to provide nested spiral trees: the idea is to draw smaller similar spirals instead of aggregating sub-trees into rectangles. For smaller sub-trees this approach will help to see the structure without lens.

For better user experience it is possible to provide animation for different events: lens appearing and hiding in the cluster graph, smooth level scrolling in the Gene Ontology visualization, step-by-step elements highlighting.