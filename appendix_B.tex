\myappendix{Subgraph Extraction Algorithm}
%\addcontentsline{toc}{section}{Appendix B: Subgraph Extraction Algorithm}
\label{sec:appendix_B}

\begin{center}
\small
\begin{lstlisting}[language=Java]
// selected vertex in the Gene Ontology by user
GO_vertex;

if GO_vertex is in cache {
  // already computed, load from cache
  load from cache GO_subgraph and Cluster_subgraph;
  return;
}

// extract subgraph using non-recursive DFS
// starting from vertex GO_vertex as root
GO_subgraph = extractSubgraph(GO_graph, GO_vertex);
save GO_subgraph to cache with key GO_vertex

// create empty subgraph
Cluster_subtree = new Graph;

for each vertex in GO_subgraph {
  if vertex is leaf {
    // leaf labels are the same for both graphs,
    // but vertex objects are different
    l_label = GO_graph.getLabel(vertex);
    leaf = Cluster_graph.getVertexByLabel(l_label);
    // get all connected vertices
    // from current leaf up to the Cluster root
    connectedVertices = invertDFS(Cluster_graph, leaf);

    // add connected vertices to cluster subtree
    // create edges
    for (int i=0; i<=connectedVertices.size()-1; i++) {
      node1, node2 = null;
      node1 = connectedNodes.get(i);
      Cluster_subtree.addVertex(node1);

      if (i+1<=connectedNodes.size()-1) {
      node2 = connectedNodes.get(i+1);

      Cluster_subtree.addVertex(node2);
      Cluster_subtree.addEdge(node2,node1);
    }
  }

  // find lowest common root for subgraph
  // and remove vertices from the root to common root
  removeRootChain(Cluster_graph, Cluster_subtree);
}

save Cluster_subtree to cache with key GO_vertex
\end{lstlisting}
\end{center}