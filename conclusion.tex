\section{Conclusion and future work}
\label{sec:conclusion}

We presented a new method for the combined visualization of an ontology (represented as DAG) and an hierarchical clustering (represented as tree) of one data set. The proposed method interactively visualizes all the data without scrolling, thereby presenting an complete overview. It also allows for interactive selection and navigation to explore the data. We have showed that the program is able to tackle the problem in our research focus, i.e., the visualization and visual mapping between two huge and conceptually different data sets. However, there are some improvements that should be performed in the future.

The current state of the prototype does not provide a way to visualize a direct mapping between a terminal GO DAG node and a cluster tree leave. A simple way to overcome this problem for one specific node is to highlight the corresponding nodes in the GO view and/or Cluster Tree view on mouse-over action. This could be easily implemented as a part of our future work.

As explained previously, the zoomed-in view for the GO shows three levels at the same time while displaying the subgraph by highlighting the nodes only. The edges are omitted due to clutter problems that can occur since edges from a higher level might go through the zoomed-in view to nodes in the lower layers. Since we are zoomed in, this does no make sense to show, because we have no insight from which layer those edges are coming from, nor to which layer they are going to. However, an improvement is possible by showing only edges between the three layers shown in the zoomed-in GO view. At the same time, the edge bundling algorithm could also be improved.

We are also working on a heavily improved version of our spiral tree metaphor to cope with more balanced binary trees. One possible solution is to create something we call nested spiral trees. The idea is to draw smaller spirals instead of aggregating larger subtrees that pass over a certain threshold of nodes into box glyphs. This approach will introduce more unused spaces, making the approach less space-filling.